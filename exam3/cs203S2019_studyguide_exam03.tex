\documentclass[11pt]{article}

% NOTE: The "Edit" sections are changed for each assignment

% Edit these commands for each assignment

\newcommand{\assignmentduedate}{May 3}
\newcommand{\assignmentassignedate}{April 29}
\newcommand{\assignmentnumber}{Three}

\newcommand{\labyear}{2019}
\newcommand{\assignedday}{Monday}
\newcommand{\dueday}{Friday}
\newcommand{\labtime}{7:00 pm}

\newcommand{\assigneddate}{Announced: \assignedday, \assignmentassignedate, \labyear{}}
\newcommand{\duedate}{Exam: \dueday, \assignmentduedate, \labyear{} at \labtime{}}

% Edit these commands to give the name to the main program

\newcommand{\mainprogram}{\lstinline{DisplayOutput}}
\newcommand{\mainprogramsource}{\lstinline{src/main/java/labone/DisplayOutput.java}}

% Edit this commands to describe key deliverables

\newcommand{\reflection}{\lstinline{writing/reflection.md}}

% Commands to describe key development tasks

% --> Running gatorgrader.sh
\newcommand{\gatorgraderstart}{\command{./gatorgrader.sh --start}}
\newcommand{\gatorgradercheck}{\command{./gatorgrader.sh --check}}

% --> Compiling and running program with gradle
\newcommand{\gradlebuild}{\command{gradle build}}
\newcommand{\gradlerun}{\command{gradle run}}

% Commands to describe key git tasks

% NOTE: Could be improved, problems due to nesting

\newcommand{\gitcommitfile}[1]{\command{git commit #1}}
\newcommand{\gitaddfile}[1]{\command{git add #1}}

\newcommand{\gitadd}{\command{git add}}
\newcommand{\gitcommit}{\command{git commit}}
\newcommand{\gitpush}{\command{git push}}
\newcommand{\gitpull}{\command{git pull}}

\newcommand{\gitcommitmainprogram}{\command{git commit src/main/java/labone/DisplayOutput.java -m "Your
descriptive commit message"}}

% The titles of the new books

\newcommand{\cooperative}{{\em Cooperative Software Design\/}}
\newcommand{\philosophy}{{\em Philosophy of Software Design\/}}
\newcommand{\thinkpython}{{\em Think Python\/}}
\newcommand{\programmingstyle}{{\em Exercises in Programming Style\/}}
\newcommand{\pytest}{{\em Python Testing with Pytest\/}}

% Use this when displaying a new command

\newcommand{\command}[1]{``\lstinline{#1}''}
\newcommand{\program}[1]{\lstinline{#1}}
\newcommand{\url}[1]{\lstinline{#1}}
\newcommand{\channel}[1]{\lstinline{#1}}
\newcommand{\option}[1]{``{#1}''}
\newcommand{\step}[1]{``{#1}''}

\usepackage{pifont}
\newcommand{\checkmark}{\ding{51}}
\newcommand{\naughtmark}{\ding{55}}

\usepackage{listings}
\lstset{
  basicstyle=\small\ttfamily,
  columns=flexible,
  breaklines=true
}

\usepackage{fancyhdr}

\usepackage[margin=1in]{geometry}
\usepackage{fancyhdr}

\pagestyle{fancy}

\fancyhf{}
\rhead{Computer Science 203}
\lhead{Exam \assignmentnumber{}}
\rfoot{Page \thepage}
\lfoot{\duedate}

\usepackage{titlesec}
\titlespacing\section{0pt}{6pt plus 4pt minus 2pt}{4pt plus 2pt minus 2pt}

\newcommand{\guidetitle}[1]
{
  \begin{center}
    \begin{center}
      \bf
      CMPSC 203\\Software Engineering\\
      Fall 2019\\
      \medskip
    \end{center}
    \bf
    #1
  \end{center}
}

\begin{document}

\thispagestyle{empty}

\guidetitle{Exam \assignmentnumber{} Study Guide \\ \assigneddate{} \\ \duedate{}}

\section*{Introduction}

\noindent The exam will be ``closed notes'' and ``closed book'' and it will
cover the following materials. Please review the ``Course Schedule'' on the Web
site for the course to see the content and slides that we have covered to this
date. Students may post questions about this material to our Slack workspace.

\vspace*{-.5em}
\begin{itemize}

  \itemsep 0in

  \item Chapters 1 through 6 and 12 through 15 and 16 through 18 in \philosophy{}.

  \item Chapters 1 through 7 and 9 through 14 in \cooperative{}.

  \item Chapters 1 through 5 in \pytest{}.

  \item Selected chapters from 1 through 19 in \programmingstyle{}.

  \item All relevant chapters in \thinkpython{} (e.g., be able to write and
    document a Python program).

  \item All design, implementation, and testing details for the first long-term
    software project.

  \item Your class notes, class activities, lecture slides, and the first seven
    practical assignments.

  \item Knowledge of the basic commands necessary for using Git and GitHub;
    understanding of the ``GitHub Flow'' model for collaboration; basic
    understanding of the Markdown syntax.

  \item The concepts introduced and reviewed during the laboratory and practical
    sessions (e.g., the key programming styles and how to use testing and
    linting tools in the Python language).

  \item Lessons learned from the team-based implementation and delivery of
    real-world software.

\end{itemize}
\vspace*{-.5em}

\noindent The examination will include a mix of questions that will require you
to draw and/or comment on a diagram, write a short answer, explain and/or write
a source code segment, or give and comment on a list of concepts or points. The
emphasis will be on the following list of illustrative subjects. Please note
that this list is not exhaustive --- rather it is designed to suggest
representative topics.

When studying for the test, don't forget that the Web site for our course
contains mobile-ready slides that will provide you with an overview of the key
concepts that we discussed in the previous modules. You can use the color scheme
in the slides to notice points where we, for instance, completed an in-class
activity, discussed a key point, or made reference to additional details
available in the one of the textbooks. If students have questions about the
material on this review sheet, then they should schedule a meeting with the
instructor or ask a question in Slack.

\vspace*{-.05in}
\begin{itemize}

  \itemsep 0in
  \itemsep 0.05in

  \item Topics from \philosophy{}:
    %
    \vspace*{-.05in}
    \begin{itemize}

      \itemsep 0.1in

      \item Knowledge of the fundamental nature of and causes for software
        complexity

      \item The similarities and differences between strategic and tactical
        programming

      \item Red flags for poor module design (e.g., ``shallow module'' and
        ``information leakage'')

      \item The relationship between general purpose modules and information
        hiding

      \item Why a pass-through method is a hallmark of poor software design

      \item The common excuses given by software engineers for not writing
        source code comments

      \item Practical strategies for avoiding the red flags associated with poor
        source code comments

      \item How to apply the maxim ``what and how, not why'' to commenting a
        program's code

      \item The ability to fix code comments so that they reduce a software
        system's complexity

      \item Best practices for writing version control commit messages and source code comments

      \item Strategies and the trade-offs associated with naming and referencing program variables

      \item Knowledge of (and solutions for) all red-flags associated with poor software engineering

    \end{itemize}

  \item Topics from \cooperative{}:
    %
    \vspace*{-.05in}
    \begin{itemize}

      \itemsep 0.1in
      % \itemsep 0in

      \item The role that software engineers play in a technical organization

      \item The ways in which communication is central to the process of
        software engineering

      \item How to measure the productivity and ``technical waste'' of a
        software engineering team

      \item The criteria used to evaluate the quality of a software requirement

      \item Knowledge of the trade-offs associated with different software
        architectures

      \item How the spiral model is an alternative to the waterfall model for
        software engineering

      \item The key ideas associated with agile methods and extreme programming

      \item The similarities and differences between verification and validation
        activities

      \item Precise definitions and examples of the terms ``fault'',
        ``failure'', and ``error''

      \item The limitations and benefits of code reviews and both manual and
        automated testing

      \item How code coverage and mutation analysis support automated and manual
        testing

      \item The purposes, benefits, and drawbacks of different types of testing
        (e.g., unit testing)

      \item How program analysis and monitoring aid the creation of efficient
        and correct software

    \end{itemize}

  \item Topics from \pytest{}:
    %
    \vspace*{-.05in}
    \begin{itemize}

      \itemsep 0.1in

      \item The benefits and drawbacks associated with automatically testing a
        Python function

      \item The six possible outcomes from running a Pytest test case on a
        Python function

      \item The role that test oracles (i.e., \program{assert} calls) play
        during the execution of a test case

      \item The benefits associated with implementing and running parameterized
        test cases

      \item The need for test case independence and the way in which test
        fixtures provide it

      \item How to use Python and Pytest to implement and test large-scale
        software applications

    \end{itemize}

  \item Topics from \programmingstyle{}:
    %
    \vspace*{-.05in}
    \begin{itemize}

      \itemsep 0.1in

      \item The input, output, and behavior of and code for the program
        calculating term frequency

      \item The way in which a programming style emerges from a system's design constraints

      \item The benefits and drawbacks associated with using styles during
        software engineering

      \item How different programming styles support and/or limit various
        software testing methods

      \item The ways in which programming styles connect to red-flags for
        software engineering

      \item How to apply the studied programming styles in the Python
        programming language

      \item How to refactor a Python program so that it correctly adheres to a
        programming style

      \item The constraints and Python code associated with the following
        programming styles:

        \begin{itemize}
          \itemsep 0.025in
          \item Monolithic
          \item Cookbook
          \item Pipeline
          \item Kick Forward
          \item Things
          \item Aspects
          \item Plugins
        \end{itemize}

    \end{itemize}

\end{itemize}

\section*{Reminder Concerning the Honor Code}

\vspace*{-.025in}

\noindent Students are required to fully adhere to the Honor Code during the
completion of this exam. More details about the Allegheny College Honor Code are
provided on the syllabus. Students are strongly encouraged to carefully review
the full statement of the Honor Code before taking this exam. If you do not
understand Allegheny College's Honor Code, please schedule a meeting with the
instructor.

The following is a review of Honor Code statement from the course
syllabus:

\begin{quote}
  The Academic Honor Program that governs the entire academic program at
  Allegheny College is described in the Allegheny Academic Bulletin. The Honor
  Program applies to all work that is submitted for academic credit or to meet
  non-credit requirements for graduation at Allegheny College. This includes all
  work assigned for this class (e.g., examinations, laboratory assignments, and
  the final project). All students who have enrolled in the College will work
  under the Honor Program. Each student who has matriculated at the College has
  acknowledged the following pledge:
\end{quote}

\begin{quote}
  I hereby recognize and pledge to fulfill my responsibilities, as defined in
  the Honor Code, and to maintain the integrity of both myself and the College
  community as a whole.
\end{quote}

\end{document}
