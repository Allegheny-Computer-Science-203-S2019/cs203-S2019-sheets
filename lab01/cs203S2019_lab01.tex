\documentclass[11pt]{article}

% NOTE: The "Edit" sections are changed for each assignment

% Edit these commands for each assignment

\newcommand{\assignmentduedate}{January 23}
\newcommand{\assignmentassignedate}{January 18}
\newcommand{\assignmentnumber}{One}

\newcommand{\labyear}{2019}
\newcommand{\labdueday}{Wednesday}
\newcommand{\labassignday}{Friday}
\newcommand{\labtime}{1:30 pm}

\newcommand{\assigneddate}{Assigned: \labassignday, \assignmentassignedate, \labyear{} at \labtime{}}
\newcommand{\duedate}{Due: \labdueday, \assignmentduedate, \labyear{} at \labtime{}}

% Edit these commands to give the name to the main program

\newcommand{\mainprogram}{\lstinline{compute_tf_monolith.py}}
\newcommand{\mainprogramsource}{\lstinline{src/termfrequency/compute_tf_monolith.py}}

% Edit this commands to describe key deliverables

\newcommand{\reflection}{\lstinline{writing/reflection.md}}

% Commands to describe key development tasks

% --> Running gatorgrader.sh
\newcommand{\gatorgraderstart}{\command{gradle grade}}
\newcommand{\gatorgradercheck}{\command{gradle grade}}

% --> Compiling and running program with gradle
\newcommand{\gradlebuild}{\command{gradle build}}
\newcommand{\gradlerun}{\command{gradle run}}

% Commands to describe key git tasks

% NOTE: Could be improved, problems due to nesting

\newcommand{\gitcommitfile}[1]{\command{git commit #1}}
\newcommand{\gitaddfile}[1]{\command{git add #1}}

\newcommand{\gitadd}{\command{git add}}
\newcommand{\gitcommit}{\command{git commit}}
\newcommand{\gitpush}{\command{git push}}
\newcommand{\gitpull}{\command{git pull}}

\newcommand{\gitcommitmainprogram}{\command{git commit src/termfrequency/compute_tf_monolith.py -m "Descriptive commit message"}}

% Use this when displaying a new command

\newcommand{\command}[1]{``\lstinline{#1}''}
\newcommand{\program}[1]{\lstinline{#1}}
\newcommand{\url}[1]{\lstinline{#1}}
\newcommand{\channel}[1]{\lstinline{#1}}
\newcommand{\option}[1]{``{#1}''}
\newcommand{\step}[1]{``{#1}''}

\usepackage{pifont}
\newcommand{\checkmark}{\ding{51}}
\newcommand{\naughtmark}{\ding{55}}

\usepackage{listings}
\lstset{
  basicstyle=\small\ttfamily,
  columns=flexible,
  breaklines=true
}

\usepackage{fancyhdr}

\usepackage[margin=1in]{geometry}
\usepackage{fancyhdr}

\pagestyle{fancy}

\fancyhf{}
\rhead{Computer Science 203}
\lhead{Practical Assignment \assignmentnumber{}}
\rfoot{Page \thepage}
\lfoot{\duedate}

\usepackage{titlesec}
\titlespacing\section{0pt}{6pt plus 4pt minus 2pt}{4pt plus 2pt minus 2pt}

\newcommand{\labtitle}[1]
{
  \begin{center}
    \begin{center}
      \bf
      CMPSC 203\\Software Engineering\\
      Spring 2019\\
      \medskip
    \end{center}
    \bf
    #1
  \end{center}
}

\begin{document}

\thispagestyle{empty}

\labtitle{Practical \assignmentnumber{} \\ \assigneddate{} \\ \duedate{}}

\section*{Objectives}

To practice using GitHub to access the files for a practical assignment.
Additionally, to practice using the Ubuntu operating system and software
development programs such as a ``terminal window'' and a ``text editor''. You
will also continue to practice using Slack to support communication with the
teaching assistants and the course instructor. Next, you will learn how to
implement and run a Python program, also discovering how to use the Pipenv tool
and the course's automated grading tool to assess your progress towards
correctly completing the project. Finally, you will explore how to perform
automated testing with programs implemented in the monolithic style.

\section*{Suggestions for Success}

\begin{itemize}
  \setlength{\itemsep}{0pt}

\item {\bf Use the laboratory computers}. The computers in this laboratory
  feature specialized software for completing this course's Laboratory and
  practical assignments. If it is necessary for you to work on a different
  machine, be sure to regularly transfer your work to a laboratory machine so
  that you can check its correctness. If you cannot use a laboratory computer
  and you need help with the configuration of your own laptop, then please
  carefully explain its setup to a teaching assistant or the course instructor
  when you are asking questions.

\item {\bf Follow each step carefully}. Slowly read each sentence in the
  assignment sheet, making sure that you precisely follow each instruction.
  Take notes about each step that you attempt, recording your questions and
  ideas and the challenges that you faced. If you are stuck, then please tell a
  teaching assistant or instructor what assignment step you recently completed.

\item {\bf Regularly ask and answer questions}. Please log into Slack at the
  start of a laboratory or practical session and then join the appropriate
  channel. If you have a question about one of the steps in an assignment, then
  you can post it to the designated channel. Or, you can ask a student sitting
  next to you or talk with a teaching assistant or the course instructor.

\item {\bf Store your files in GitHub}. As in all of your past assignments, you
  will be responsible for storing all of your files (e.g., Python source code and
  Markdown-based writing) in a Git repository using GitHub Classroom. Please
  verify that you have saved your source code in your Git repository by using
  \command{git status} to ensure that everything is updated. You can see if
  your assignment submission meets the established correctness requirements by
  using the provided checking tools on your local computer and in checking the
  commits in GitHub.

\item {\bf Keep all of your files}. Don't delete your programs, output files,
  and written reports after you submit them through GitHub; you will need them
  again when you study for the quizzes and examinations and work on the other
  laboratory, practical, and final project assignments.

\item {\bf Back up your files regularly}. All of your files are regularly
  backed-up to the servers in the Department of Computer Science and, if you
  commit your files regularly, stored on GitHub. However, you may want to use a
  flash drive, Google Drive, or your favorite backup method to keep an extra
  copy of your files on reserve. In the event of any type of system failure,
  you are responsible for ensuring that you have access to a recent backup copy
  of all your files.

\item {\bf Explore teamwork and technologies}. While certain aspects of these
  assignments will be challenging for you, each part is designed to give you the
  opportunity to learn both fundamental concepts in the field of computer
  science and explore advanced technologies that are commonly employed at a wide
  variety of companies. To explore and develop new ideas, you should regularly
  communicate with your team and/or the teaching assistants and tutors.

\item {\bf Hone your technical writing skills}. Computer science assignments
  require to you write technical documentation and descriptions of your
  experiences when completing each task. Take extra care to ensure that your
  writing is interesting and both grammatically and technically correct,
  remembering that computer scientists must effectively communicate and
  collaborate with their team members and the tutors, teaching assistants, and
  course instructor.

\item {\bf Review the Honor Code on the syllabus}. While you may discuss your
  assignments with others, copying source code or writing is a violation of
  Allegheny College's Honor Code.

\end{itemize}

\section*{Reading Assignment}

If you have not done so already, please read all of the relevant ``GitHub
Guides'', available at \url{https://guides.github.com/}, that explain how to use
many of the features that GitHub provides. In particular, please make sure that
you have read guides such as ``Mastering Markdown'' and ``Documenting Your
Projects on GitHub''; each of them will help you to understand how to use both
GitHub and GitHub Classroom. To do well on this assignment, you should also read
Chapters 1 and 2 in {\em Think Python\/} and the Preface and Chapters 1 through
3 in the {\em Exercises in Programming Style\/}.
%
You are also expected to find and read all of the online resources that
you need to complete this assignment.
%
Please see the course instructor if you have questions on these reading
assignments.

\section*{Debugging and Testing a Monolithic Python Program}

To access the practical assignment, you should go into the
\channel{\#announcements} channel in our Slack team and find the announcement
that provides a link for it. Copy this link and paste it into your web browser.
Now, you should accept the practical assignment and see that GitHub Classroom
created a new GitHub repository for you to access the assignment's starting
materials and to store the completed version of your assignment. Specifically,
to access your new GitHub repository for this assignment, please click the green
``Accept'' button and then click the link that is prefaced with the label ``Your
assignment has been created here''. If you accepted the assignment and correctly
followed these steps, you should have created a GitHub repository with a name
like
``Allegheny-Computer-Science-203-Spring-2019/computer-science-203-spring-2019-practical-1-gkapfham''.
Unless you provide the instructor with documentation of the extenuating
circumstances that you are facing, not accepting the assignment means that you
automatically receive a failing grade for it. Now, remember that your home base
for this assignment is the directory that contains all of the configuration
files and the \command{src/} directory. Make sure that you can find your ``home
base'' for this practical assignment! Please see the instructor if you are stuck
on getting started.

Now, study the provided Python source code and the technical documentation to
understand the type of output that your program should produce.
%
At the outset, you should notice that the provided source code does not contain
all of the source code from Chapter 3 of the book. You will need to add in the
appropriate source code and documentation to ensure that the Python program
passes all of the checks and produces the correct output. Make sure that you
fully document all aspects of the Python program! Next, you need to implement
and explain some approach to automatically testing this monolithic Python
program. Your approach should be executable at the command-line and always
ensure that the program produces the expected output for both of the two
provided input files. Please explain your automated testing strategy in
\reflection{}.

\section*{Checking the Correctness of Your Program and Writing}

In addition to writing your own automated testing tool, you can use GatorGrader
to check other characteristics of your project. To automatically check your
Python source code you can get started with the use of the GatorGrader tool,
typing the command \gatorgraderstart{} in your terminal window.
%
Once your program is running correctly, fulfilling at least some of the
assignment's requirements, you should transfer your files to GitHub using the
\gitcommit{} and \gitpush{} commands. For example, if you want to signal that
the \mainprogramsource{} file has been changed and is ready for transfer to
GitHub you would first type \gitcommitmainprogram{} in your terminal, followed
by typing \gitpush{} and checking to see that the transfer to GitHub is
successful. If you notice that transferring your Python source code to GitHub
did not work correctly, then please try to determine why, asking a teaching
assistant or the course instructor for assistance, if necessary.

% If you do have mistakes in your assignment, then you will need to review
% GatorGrader's output, find the mistake, and try to fix it.

After the course instructor enables \step{continuous integration} with a system
called Travis CI, when you use the \gitpush{} command to transfer your source
code to your GitHub repository, Travis CI will initialize a \step{build} of your
assignment, checking to see if it meets all of the requirements. If both your
source code and writing meet all of the established requirements, then you will
see a green \checkmark{} in the listing of commits in GitHub after awhile. If
your submission does not meet the requirements, a red \naughtmark{} will appear
instead. The instructor will reduce a student's grade for this assignment if the
red \naughtmark{} appears on the last commit in GitHub immediately before the
assignment's due date. Yet, if the green \checkmark{} appears on the last commit
in your GitHub repository, then you satisfied all of the main checks and you
will earn full credit. Unless you provide the course instructor with
documentation of the severe and extenuating circumstances that you are facing,
no late work will be considered towards your completion grade for this practical
assignment.

\section*{Summary of the Required Deliverables}

\noindent Students do not need to submit printed source code or technical
writing for any assignment in this course. Instead, this assignment invites you
to submit, using GitHub, the following deliverables.

% Because this is a practical assignment, you are not required to complete any
% technical writing.

\begin{enumerate}

\setlength{\itemsep}{0in}

\item A properly documented, well-formatted, and correct version of
  \mainprogramsource{} that both meets all of the established requirements and
  produces the desired output.

\item Stored in a Markdown file called \reflection{}, a four-paragraph response,
  with each paragraph consisting of at least 100 words, for each of the stated
  prompts.

\end{enumerate}

\section*{Evaluation of Your Practical Assignment}

Practical assignments are graded on a completion --- or ``checkmark'' --- basis. If
your GitHub repository has a \checkmark{} for the last commit before the
deadline then you will receive the highest possible grade for the assignment.
However, you will fail the assignment if you do not complete it correctly, as
evidenced by a red \naughtmark{} in your commit listing, by the set deadline for
completing the project.

\section*{Adhering to the Honor Code}

In adherence to the Honor Code, students should complete this assignment on an
individual basis. While it is appropriate for students in this class to have
high-level conversations about the assignment, it is necessary to distinguish
carefully between the student who discusses the principles underlying a problem
with others and the student who produces assignments that are identical to, or
merely variations on, someone else's work. Deliverables (e.g., Python source
code) that are nearly identical to the work of others will be taken as evidence
of violating the \mbox{Honor Code}. Please see the course instructor during
office hours if you have questions about the Honor Code policy.

\end{document}
