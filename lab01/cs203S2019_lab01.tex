\documentclass[11pt]{article}

% NOTE: The "Edit" sections are changed for each assignment

% Edit these commands for each assignment

\newcommand{\assignmentduedate}{February 1}
\newcommand{\assignmentassignedate}{January 22}
\newcommand{\assignmentnumber}{One}

\newcommand{\labyear}{2019}
\newcommand{\labdueday}{Friday}
\newcommand{\labassignday}{Tuesday}
\newcommand{\labtime}{2:30 pm}

\newcommand{\assigneddate}{Assigned: \labassignday, \assignmentassignedate, \labyear{} at \labtime{}}
\newcommand{\duedate}{Due: \labdueday, \assignmentduedate, \labyear{} at \labtime{}}

% NOTE: there was no source code for this assignment
% Instead, the focus was on two Markdown files

\newcommand{\assessment}{\lstinline{assessment.md}}
\newcommand{\conduct}{\lstinline{conduct.md}}

% Edit these commands to give the name to the main program

\newcommand{\mainprogram}{\lstinline{compute_tf_monolith.py}}
\newcommand{\mainprogramsource}{\lstinline{src/termfrequency/compute_tf_monolith.py}}

% Edit this commands to describe key deliverables

\newcommand{\reflection}{\lstinline{report.md}}

% Commands to describe key development tasks

% --> Running gatorgrader.sh
\newcommand{\gatorgraderstart}{\command{gradle grade}}
\newcommand{\gatorgradercheck}{\command{gradle grade}}

% --> Compiling and running program with gradle
\newcommand{\gradlebuild}{\command{gradle build}}
\newcommand{\gradlerun}{\command{gradle run}}

% Commands to describe key git tasks

% NOTE: Could be improved, problems due to nesting

\newcommand{\gitcommitfile}[1]{\command{git commit #1}}
\newcommand{\gitaddfile}[1]{\command{git add #1}}

\newcommand{\gitadd}{\command{git add}}
\newcommand{\gitcommit}{\command{git commit}}
\newcommand{\gitpush}{\command{git push}}
\newcommand{\gitpull}{\command{git pull}}

\newcommand{\gitcommitmainprogram}{\command{git commit src/termfrequency/compute_tf_monolith.py -m "Descriptive commit message"}}

% Commands for the textbooks, since there are so many

\newcommand{\cooperative}{{\em Cooperative Software Design\/}}
\newcommand{\philosophy}{{\em Philosophy of Software Design\/}}
\newcommand{\thinkpython}{{\em Think Python\/}}
\newcommand{\programmingstyle}{{\em Exercises in Programming Style\/}}
\newcommand{\pytest}{{\em Python Testing with Pytest\/}}

% Use this when displaying a new command

\newcommand{\command}[1]{``\lstinline{#1}''}
\newcommand{\program}[1]{\lstinline{#1}}
\newcommand{\url}[1]{\lstinline{#1}}
\newcommand{\channel}[1]{\lstinline{#1}}
\newcommand{\option}[1]{``{#1}''}
\newcommand{\step}[1]{``{#1}''}

\usepackage{pifont}
\newcommand{\checkmark}{\ding{51}}
\newcommand{\naughtmark}{\ding{55}}

\usepackage{listings}
\lstset{
  basicstyle=\small\ttfamily,
  columns=flexible,
  breaklines=true
}

\usepackage{fancyhdr}

\usepackage[margin=1in]{geometry}
\usepackage{fancyhdr}

\pagestyle{fancy}

\fancyhf{}
\rhead{Computer Science 203}
\lhead{Laboratory Assignment \assignmentnumber{}}
\rfoot{Page \thepage}
\lfoot{\duedate}

\usepackage{titlesec}
\titlespacing\section{0pt}{6pt plus 4pt minus 2pt}{4pt plus 2pt minus 2pt}

\newcommand{\labtitle}[1]
{
  \begin{center}
    \begin{center}
      \bf
      CMPSC 203\\Software Engineering\\
      Spring 2019\\
      \medskip
    \end{center}
    \bf
    #1
  \end{center}
}

\begin{document}

\thispagestyle{empty}

\labtitle{Laboratory \assignmentnumber{} \\ \assigneddate{} \\ \duedate{}}

\section*{Objectives}

To learn how to use GitHub and the GitHub flow model to support collaboration
among a team of software engineers. Along with exploring the use of GitHub
features like the issue tracker and the pull request reviewer, in this
assignment you will learn how to use Markdown to complete technical writing
tasks. As a side effect of working in a team, you will also experience
challenges, such as the creation of merge conflicts in a version control
repository, that force you to develop practical solutions. You will also gain
experience in talking with team members and leaders, technical leaders, and the
course instructor. Ultimately, students will work together to write (i) an
assessment guide to evaluate a team member's mastery of technical and
professional skills and (ii) a code of conduct governing how all team members
will interact during the completion of software products.

\section*{Suggestions for Success}

\begin{itemize}
  \setlength{\itemsep}{0pt}

\item {\bf Use the laboratory computers}. The computers in this laboratory
  feature specialized software for completing this course's Laboratory and
  practical assignments. If it is necessary for you to work on a different
  machine, be sure to regularly transfer your work to a laboratory machine so
  that you can check its correctness. If you cannot use a laboratory computer
  and you need help with the configuration of your own laptop, then please
  carefully explain its setup to a teaching assistant or the course instructor
  when you are asking questions.

\item {\bf Follow each step carefully}. Slowly read each sentence in the
  assignment sheet, making sure that you precisely follow each instruction.
  Take notes about each step that you attempt, recording your questions and
  ideas and the challenges that you faced. If you are stuck, then please tell a
  teaching assistant or instructor what assignment step you recently completed.

\item {\bf Regularly ask and answer questions}. Please log into Slack at the
  start of a laboratory or practical session and then join the appropriate
  channel. If you have a question about one of the steps in an assignment, then
  you can post it to the designated channel. Or, you can ask a student sitting
  next to you or talk with a teaching assistant or the course instructor.

\item {\bf Store your files in GitHub}. As in all of your past assignments, you
  will be responsible for storing all of your files (e.g., Python source code and
  Markdown-based writing) in a Git repository using GitHub Classroom. Please
  verify that you have saved your source code in your Git repository by using
  \command{git status} to ensure that everything is updated. You can see if
  your assignment submission meets the established correctness requirements by
  using the provided checking tools on your local computer and in checking the
  commits in GitHub.

\item {\bf Keep all of your files}. Don't delete your programs, output files,
  and written reports after you submit them through GitHub; you will need them
  again when you study for the quizzes and examinations and work on the other
  laboratory, practical, and final project assignments.

\item {\bf Back up your files regularly}. All of your files are regularly
  backed-up to the servers in the Department of Computer Science and, if you
  commit your files regularly, stored on GitHub. However, you may want to use a
  flash drive, Google Drive, or your favorite backup method to keep an extra
  copy of your files on reserve. In the event of any type of system failure,
  you are responsible for ensuring that you have access to a recent backup copy
  of all your files.

\item {\bf Explore teamwork and technologies}. While certain aspects of these
  assignments will be challenging for you, each part is designed to give you the
  opportunity to learn both fundamental concepts in the field of computer
  science and explore advanced technologies that are commonly employed at a wide
  variety of companies. To explore and develop new ideas, you should regularly
  communicate with your team and/or the teaching assistants and tutors.

\item {\bf Hone your technical writing skills}. Computer science assignments
  require to you write technical documentation and descriptions of your
  experiences when completing each task. Take extra care to ensure that your
  writing is interesting and both grammatically and technically correct,
  remembering that computer scientists must effectively communicate and
  collaborate with their team members and the tutors, teaching assistants, and
  course instructor.

\item {\bf Review the Honor Code on the syllabus}. While you may discuss your
  assignments with others, copying source code or writing is a violation of
  Allegheny College's Honor Code.

\end{itemize}

\section*{Reading Assignment}

% Module One Reading Assignment:

% Cooperative Software Design, Chapters 1 - 3
% Philosophy of Software Design, Chapters 1 - 3
% Think Python, Chapters 1 - 3
% Exercises in Programming Style, Prologue, Preface, Chapters 1 - 4
% Python Testing with Pytest, Preface, Chapters 1 - 2

If you have not done so already, please read all of the relevant ``GitHub
Guides'', available at \url{https://guides.github.com/}, that explain how to use
many of the features that GitHub provides. In particular, please make sure that
you have read guides such as ``Mastering Markdown'' and ``Documenting Your
Projects on GitHub''; each of them will help you to understand how to use both
GitHub and GitHub Classroom.
%
To do well on this assignment you should also read Chapters 1 through 3 in in
both \cooperative{} and \philosophy{}.
%
You are also expected to find and read all of the online resources that you need
to complete this laboratory assignment.
%
Please see the course instructor if you have questions on these reading
assignments.

\section*{Creating an Assessment Guide and Code of Conduct}

% Accept the assignment

To access the practical assignment, you should go into the
\channel{\#announcements} channel in our Slack team and find the announcement
that provides a link for it. Copy this link and paste it into your web browser.
Now, you should accept the laboratory assignment and see that GitHub Classroom
created a new GitHub repository for you to access the assignment's starting
materials and to store the completed version of your assignment. Specifically,
to access your new GitHub repository for this assignment, please click the green
``Accept'' button and then click the link that is prefaced with the label ``Your
assignment has been created here''. If you accepted the assignment and correctly
followed these steps, you should have created a GitHub repository with a name
like
``Allegheny-Computer-Science-203-Spring-2019/computer-science-203-spring-2019-lab-1-gkapfham''.
Unless you provide the instructor with documentation of the extenuating
circumstances that you are facing, not accepting the assignment means that you
will receive a failing grade for it.

% Details of what students must complete

Unlike the practical assignments, this laboratory assignment asks you to work in
a entire-class team to write two documents that are each stored in a separate
GitHub repository. If you look in the repository that GitHub Classroom created,
then you will find the link to a repository that stores the starting version of
a guide supporting the assessment of a student's technical and professional
skills. Similarly, you should also find the link to another repository that
stores the start of a code of conduct for the members of a software engineering
team. For this laboratory assignment, your task is to collaborate with the
members of your class to write a full-featured assessment guide and code of
conduct, using the GitHub flow model to facilitate collaboration. Please note
that, since you do not have direct access to either of the aforementioned
repositories, you should use repository forks and pull requests to ensure that
your work is ultimately added to the repository's master branch.

\section*{Collaborating with Your Software Engineering Team}

Your team should use GitHub and its features (e.g., issue tracker, pull
requests, commit log, and code review request) to complete all of the tasks
referenced in the previous section.
%
Aiming to manage risk and estimate the effort required for individual team
members to complete this project, you should assign people to teams, roles, and
tasks. While it is acceptable for you to have in-person discussions with your
team members or to talk about the project through Slack, please remember that
all important discussions and decisions must be documented through GitHub.
Finally, as you are working with your team, you should carefully document your
experiences and contributions so that you can share them through writing stored
in the repository created by GitHub Classroom.
%
Whenever possible, you should describe the your efforts on this laboratory
assignment according to the following levels: N = None, I = Inadequate, A =
Adequate, G = Good, and E = Excellent. Your evaluation of your own work should
focus on your mastery of both technical and professional skills in software
engineering. You should thoughtfully reflect on your current areas of expertise
and opportunities for improvement. Please share your questions about this task
with the instructor.

Since multiple approaches may support the effective completion of the required
documents, this assignment does not dictate team organization or communication
strategies. The students in the course should instead work with each other, the
technical leads, and the course instructor to identify team roles and strategies
for effective organization and communication. With that said, you should plan to
use either forks or branches of a GitHub repository to organize your work.
%
Once a specific branch contains the finished version of its associated
deliverable, a team member should create a pull request for discussion and
review. If the team leaders, the technical leaders, and the course instructor
judge that the pull request has all of the expected characteristics, then it
should be merged into the ``master'' branch of the appropriate repository. If
the pull request is not accepted, then the team member should improve it until
it meets every reviewer's expectations. Your team should continue to use this
model, called ``GitHub flow'', to support the completion of all deliverables.
%
Students with questions about the use of GitHub should first talk to a team
leader.

Please remember that Travis CI is configured to check the Markdown files in the
two repositories with both \command{mdl} and \command{proselint}.
%
If your writing meets all of the established requirements set by these linting
tools, then you will see a green \checkmark{} in the listing of commits in
GitHub after awhile. If your submission does not meet the requirements, a red
\naughtmark{} will appear instead. The instructor will reduce a student's grade
for this assignment if the red \naughtmark{} appears on the last commit in
GitHub immediately before the assignment's due date. Yet, if the green
\checkmark{} appears on the last commit in your GitHub repository, then you
satisfied all of the main checks and you will earn full credit.

\section*{Summary of the Required Deliverables}

\noindent Students do not need to submit printed source code or technical
writing for any assignment in this course. Instead, this assignment invites you
to submit, using GitHub, the following deliverables. You work will be graded on
a checkmark basis for this laboratory assignment.
%
Unless you provide the course instructor with documentation of the severe and
extenuating circumstances that you are facing, no late work will be considered
towards your completion grade for this practical assignment.

\begin{enumerate}

\setlength{\itemsep}{0in}

\item A properly documented, well-formatted, and correct version of
  \assessment{} that meets all of the established requirements and was
  collaboratively developed by the entire class.

\item A properly documented, well-formatted, and correct version of
  \conduct{} that meets all of the established requirements and was
  collaboratively developed by the entire class.

\item Stored in a Markdown file called \reflection{}, a multiple-paragraph
  response that documents and evaluates the work that you completed for this
  laboratory assignment.

\end{enumerate}

\end{document}
