\documentclass[11pt]{article}

% NOTE: The "Edit" sections are changed for each assignment

% Edit these commands for each assignment

\newcommand{\assignmentduedate}{May 3}
\newcommand{\assignmentassignedate}{March 26}
\newcommand{\assignmentnumber}{Two}

\newcommand{\labassignmentnumberstart}{Eight}
\newcommand{\labassignmentnumberend}{Twelve}

\newcommand{\labyear}{2019}
\newcommand{\labdueday}{Friday}
\newcommand{\labassignday}{Tuesday}
\newcommand{\labtime}{2:30 pm}
\newcommand{\labduetime}{12:00 midnight}

\newcommand{\assigneddate}{Assigned: \labassignday, \assignmentassignedate, \labyear{} at \labtime{}}
\newcommand{\duedate}{Due: \labdueday, \assignmentduedate, \labyear{} at \labduetime{}}

% NOTE: there was no source code for this assignment
% Instead, the focus was on two Markdown files

\newcommand{\assessment}{\lstinline{assessment.md}}
\newcommand{\conduct}{\lstinline{conduct.md}}

% Edit these commands to give the name to the main program

\newcommand{\mainprogram}{\lstinline{compute_tf_monolith.py}}
\newcommand{\mainprogramsource}{\lstinline{src/termfrequency/compute_tf_monolith.py}}

% Edit this commands to describe key deliverables

\newcommand{\reflection}{\lstinline{report.md}}

% Commands to describe key development tasks

% --> Running gatorgrader.sh
\newcommand{\gatorgraderstart}{\command{gradle grade}}
\newcommand{\gatorgradercheck}{\command{gradle grade}}

% --> Compiling and running program with gradle
\newcommand{\gradlebuild}{\command{gradle build}}
\newcommand{\gradlerun}{\command{gradle run}}

% Commands to describe key git tasks

% NOTE: Could be improved, problems due to nesting

\newcommand{\gitcommitfile}[1]{\command{git commit #1}}
\newcommand{\gitaddfile}[1]{\command{git add #1}}

\newcommand{\gitadd}{\command{git add}}
\newcommand{\gitcommit}{\command{git commit}}
\newcommand{\gitpush}{\command{git push}}
\newcommand{\gitpull}{\command{git pull}}

\newcommand{\gitcommitmainprogram}{\command{git commit src/termfrequency/compute_tf_monolith.py -m "Descriptive commit message"}}

% Commands for the textbooks, since there are so many

\newcommand{\cooperative}{{\em Cooperative Software Design\/}}
\newcommand{\philosophy}{{\em Philosophy of Software Design\/}}
\newcommand{\thinkpython}{{\em Think Python\/}}
\newcommand{\programmingstyle}{{\em Exercises in Programming Style\/}}
\newcommand{\pytest}{{\em Python Testing with Pytest\/}}

% Use this when displaying a new command

\newcommand{\command}[1]{``\lstinline{#1}''}
\newcommand{\program}[1]{\lstinline{#1}}
\newcommand{\url}[1]{\lstinline{#1}}
\newcommand{\channel}[1]{\lstinline{#1}}
\newcommand{\option}[1]{``{#1}''}
\newcommand{\step}[1]{``{#1}''}

\usepackage{pifont}
\newcommand{\checkmark}{\ding{51}}
\newcommand{\naughtmark}{\ding{55}}

\usepackage{listings}
\lstset{
  basicstyle=\small\ttfamily,
  columns=flexible,
  breaklines=true
}

\usepackage{fancyhdr}

\usepackage[margin=1in]{geometry}
\usepackage{fancyhdr}

\pagestyle{fancy}

\fancyhf{}
\rhead{Computer Science 203}

\lhead{Software Project \assignmentnumber{} (Laboratory Assignments \labassignmentnumberstart{} through \labassignmentnumberend{})}

\rfoot{Page \thepage}
\lfoot{\duedate}

\usepackage{titlesec}
\titlespacing\section{0pt}{6pt plus 4pt minus 2pt}{4pt plus 2pt minus 2pt}

\newcommand{\labtitle}[1]
{
  \begin{center}
    \begin{center}
      \bf
      CMPSC 203\\Software Engineering\\
      Spring 2019\\
      \medskip
    \end{center}
    \bf
    #1
  \end{center}
}

\begin{document}

\thispagestyle{empty}

\labtitle{Software Project \assignmentnumber{} \\ Laboratory Assignments \labassignmentnumberstart{} through \labassignmentnumberend{} \\ \assigneddate{} \\ \duedate{}}

\section*{Objectives}

To use GitHub and the GitHub flow model to collaboratively engineer, deliver,
and evaluate a software product.
%
Along with using GitHub features like the issue tracker and reviewing pull
requests, in this assignment you will use Markdown to complete technical writing
tasks and the Python programming language and many Python packages (e.g., Pytest
for automated testing) to implement a production quality program.
%
As a side effect of working in a team, you will also experience challenges, such
as the creation of merge conflicts in a version control repository, that force
you to develop practical solutions.
%
You will also gain experience in talking with team members and leaders,
technical leaders, and the course instructor.
%
Students will work together in a development team while mastering the technical
and professional skills in the field of software engineering, completing a
weekly assessment of their progress.
%
Students will also adhere to a code of conduct that governs how all team members
will interact during the completion of software products.
%
Students will regularly receive both letter and completion grades on this
software project.

\section*{Suggestions for Success}

\begin{itemize}
  \setlength{\itemsep}{0pt}

\item {\bf Use the laboratory computers}. The computers in this laboratory
  feature specialized software for completing this course's Laboratory and
  practical assignments. If it is necessary for you to work on a different
  machine, be sure to regularly transfer your work to a laboratory machine so
  that you can check its correctness. If you cannot use a laboratory computer
  and you need help with the configuration of your own laptop, then please
  carefully explain its setup to a teaching assistant or the course instructor
  when you are asking questions.

\item {\bf Follow each step carefully}. Slowly read each sentence in the
  assignment sheet, making sure that you precisely follow each instruction.
  Take notes about each step that you attempt, recording your questions and
  ideas and the challenges that you faced. If you are stuck, then please tell a
  teaching assistant or instructor what assignment step you recently completed.

\item {\bf Regularly ask and answer questions}. Please log into Slack at the
  start of a laboratory or practical session and then join the appropriate
  channel. If you have a question about one of the steps in an assignment, then
  you can post it to the designated channel. Or, you can ask a student sitting
  next to you or talk with a teaching assistant or the course instructor.

\item {\bf Store your files in GitHub}. As in all of your past assignments, you
  will be responsible for storing all of your files (e.g., Python source code and
  Markdown-based writing) in a Git repository using GitHub Classroom. Please
  verify that you have saved your source code in your Git repository by using
  \command{git status} to ensure that everything is updated. You can see if
  your assignment submission meets the established correctness requirements by
  using the provided checking tools on your local computer and in checking the
  commits in GitHub.

\item {\bf Keep all of your files}. Don't delete your programs, output files,
  and written reports after you submit them through GitHub; you will need them
  again when you study for the quizzes and examinations and work on the other
  laboratory, practical, and final project assignments.

\item {\bf Back up your files regularly}. All of your files are regularly
  backed-up to the servers in the Department of Computer Science and, if you
  commit your files regularly, stored on GitHub. However, you may want to use a
  flash drive, Google Drive, or your favorite backup method to keep an extra
  copy of your files on reserve. In the event of any type of system failure,
  you are responsible for ensuring that you have access to a recent backup copy
  of all your files.

\item {\bf Explore teamwork and technologies}. While certain aspects of these
  assignments will be challenging for you, each part is designed to give you the
  opportunity to learn both fundamental concepts in the field of computer
  science and explore advanced technologies that are commonly employed at a wide
  variety of companies. To explore and develop new ideas, you should regularly
  communicate with your team and/or the teaching assistants and tutors.

\item {\bf Hone your technical writing skills}. Computer science assignments
  require to you write technical documentation and descriptions of your
  experiences when completing each task. Take extra care to ensure that your
  writing is interesting and both grammatically and technically correct,
  remembering that computer scientists must effectively communicate and
  collaborate with their team members and the tutors, teaching assistants, and
  course instructor.

% \item {\bf Review the Honor Code on the syllabus}. While you may discuss your
%   assignments with others, copying source code or writing is a violation of
%   Allegheny College's Honor Code.

\end{itemize}

\vspace*{-.1in}
\section*{Reading Assignment}

% Course Textbooks:

% Cooperative Software Design
% Philosophy of Software Design
% Think Python
% Exercises in Programming Style
% Python Testing with Pytest

If you have not done so already, please read all of the relevant ``GitHub
Guides'', available at \url{https://guides.github.com/}, that explain how to use
many of the features that GitHub provides. In particular, please make sure that
you have read guides such as ``Mastering Markdown'' and ``Documenting Your
Projects on GitHub''; each of them will help you to understand how to use both
GitHub and GitHub Classroom.
%
To do well on this assignment you should also read all of the assigned chapters
in the following textbooks: \cooperative, \philosophy, \thinkpython,
\programmingstyle, and \pytest.
%
You are also expected to find and read all of the online resources that you need
to complete this software project.
%
Please see the course instructor if you have questions on these reading
assignments.

\section*{Developing and Releasing a Suite of Pytest Plugins}

% Accept the assignment

To access this laboratory assignment, you should go into the
\channel{\#announcements} channel in our Slack team and find the announcement
that provides a link for it. Copy this link and paste it into your web browser.
Now, you should accept the laboratory assignment and see that GitHub Classroom
created a new GitHub repository for you to access the assignment's starting
materials and to store the completed version of your assignment. Specifically,
to access your new GitHub repository for this assignment, please click the green
``Accept'' button and then click the link that is prefaced with the label ``Your
assignment has been created here''. If you accepted the assignment and correctly
followed these steps, you should have created a GitHub repository with a name
like
``Allegheny-Computer-Science-203-Spring-2019/computer-science-203-spring-2019-lab-8-gkapfham''.
Unless you provide the instructor with documentation of the extenuating
circumstances that you are facing, not accepting the assignment means that you
will receive a failing grade for it.
%
Instead of giving you access to ``starter'' code for this assignment, the
purpose of this GitHub repository is to facilitate the assessment of your own
mastery of the professional and technical skills in software engineering and to
attest to the fact that you adhered to the course's code of conduct.
%
Students must to fully complete the materials in these repositories for each
week of the project.

% Details of what students must complete

For this sofware project, your task is to collaborate with the members of your
class to design, implement, test, deploy, and maintain a suite of seven Pytest
plugins that are implemented in the Python programming language.
%
Each small team that is a part of the class-wide team must implement their own
Pytest plugin.
%
That is, everyone is responsible for working together to handle all project
issues during the completion of this long-term software project that will result
in a suite of Pytest plugins hosted in a GitHub organization.
%
At the start of this project, you should collaboratively decide what seven
plugins you will implement, leveraging your prior experiences with Pytest to
inform your decisions. For instance, it would be useful to have a Pytest plugin
that only runs the tests that focus on the modified parts of a Python program.
Your team should also consider implementing a plugin that reorders a test suite
so that those tests most likely achieve high coverage or find a bug are run
first. Another reordering plugin might run the test suite so that the fastest
tests are run first. You could also consider implementing a plugin that uses
statistics to localize the defect in the program given the history of the test
runs and the test cases that fail.
%
The student software engineers should work with the technical leads and the
course instructors to identify seven distinct Pytest plugins that they can
feasibily implement by this project's due date.

\section*{Collaborating with Your Software Engineering Team}

% Use of GitHub

Your team should use GitHub and its features (e.g., issue tracker, pull
requests, commit log, and code review request) to complete all of the tasks
referenced in the previous section.
%
Aiming to manage risk and estimate the effort required for individual team
members to complete this project, you should assign people to teams, roles, and
tasks. While it is acceptable for you to have in-person discussions with your
team members or to talk about the project through Slack, please remember that
all important discussions and decisions must be documented through GitHub.
Finally, as you are working with your team, you should carefully document your
experiences and contributions so that you can share them through writing stored
in the repository created by GitHub Classroom.

Since multiple approaches may support the effective completion of the required
software, this assignment does not dictate team organization or communication
strategies. The students in the course should instead work with each other, the
technical leads, and the course instructor to identify team roles and strategies
for effective organization and communication. With that said, you should plan to
use either forks or branches of a GitHub repository to organize your work.
%
Once a specific branch/fork contains the finished version of its associated
deliverable, a team member should create a pull request for discussion and
review. If the team leaders, the technical leaders, and the course instructor
judge that the pull request has all of the expected characteristics, then it
should be merged into the ``master'' branch of the appropriate repository. If
the pull request is not accepted, then team member(s) should improve it until it
meets every reviewer's expectations. Your team should continue to use this
model, called ``GitHub flow'', to support the completion of all deliverables.
%
Students with questions about the use of GitHub should first talk to a team
leader.

\section*{Skill Assessment and Code of Conduct}

% Ratings and code of conduct

Your GitHub repository for each week will contain links to the assessment sheet
and code of conduct that we jointly created at the start of the semester. Please
copy those files into your repository and complete them fully. For each week of
this project you should describe your efforts on this software project according
to the following levels: N = None, I = Inadequate, A = Adequate, G = Good, and E
= Excellent. Your evaluation of your own work should focus on your mastery of
both technical and professional skills in software engineering. You should
thoughtfully reflect on your current areas of expertise and opportunities for
improvement. By the mid-way and end points of this project, you should have
demonstrated mastery of at least half and all the skills in the assessment
sheet, respectively. You should also use the skill assessment sheet to document
the work that you completed for this project. As you work on this project, you
must be proactive in finding ways to master all of the listed skills. Finally,
you should ``sign'' your name in the code of conduct to indicate that you
adhered to it throughout the completion of the software project.

% Checking of the assessment sheet and the code of conduct guide

Please remember that Travis CI is configured to use both \command{mdl} and
\command{proselint} to check the Markdown files in your repository created by
GitHub Classroom.
%
If you saved the files correctly and your writing meets all of the requirements
set by these linting tools, then you will see a green \checkmark{} in the
listing of commits in GitHub after awhile. If your submission does not meet the
requirements, a red \naughtmark{} will appear instead. The instructor will
reduce a student's grade for this assignment if the red \naughtmark{} appears on
the last commit in GitHub immediately before the assignment's due date. Yet, if
the green \checkmark{} appears on the last commit in your GitHub repository,
then you satisfied the basic linting checks for the Markdown files that contain
the assessment sheet and the code of conduct.

\section*{Suggested Schedule for the Software Project}

The course instructor invites the students in this class to work together to
devise a schedule by which they can complete the software product by the stated
deadline. Overall, you will work on this assignment for five weeks. Here is a
suggestion for a schedule to complete the plugin suite:

\begin{itemize}

  \setlength{\itemsep}{0pt}

  \item {\bf Week One}: Along with deciding which seven Pytest plugins you will
    implement, you should use the GatorGrouper tool to create the seven teams
    that will each create a functional plugin in a GitHub repository. After
    learning how to build, test, and use the sample Pytest plugin provided in
    the textbook, create a GitHub organization for the suite of plugins, create
    repositories for each of the seven plugins, organize these repositories,
    raise issues in GitHub for key tasks to complete, organize your assigned
    engineering teams, and assign tasks to individuals.

  \item {\bf Week Two}: Finish implementing prototype features and give a
    demonstration to people who will use your plugins, requesting feedback that
    you incorporate into subsequent versions.

  \item {\bf Week Three}: Implement all key features and complete all major
    fixes, ultimately leading to a full-featured demonstration and further
    feedback from more people who use your plugins.

  \item {\bf Week Four}: While continuing to incrementally enhance each plugin,
    release a completed version of the suite so that external individuals can
    use it and provide detailed feedback through GitHub's issue tracker. After
    improving your plugin's documentation, make sure that each team can
    download, install, and use the plugins developed by the other teams.

  \item {\bf Week Five}: Release a production-quality suite of Pytest plugins
    for use by other student engineers and by the instructor, who will use it
    during the next software engineering course.

\end{itemize}

\noindent You will receive a letter grade for your cumulative work on weeks
three and five; the other interim work will be assessed on a completion basis.
Please see the instructor with your grading questions.

\section*{Summary of the Required Deliverables}

\noindent Students do not need to submit printed source code or technical
writing for any assignment in this course. Instead, this assignment invites you
to submit, using GitHub, the following deliverables. Depending on the week, your
work will be graded on a either a letter or a checkmark basis.
%
Unless you provide the instructor with documentation of the severe and
extenuating circumstances that you are facing, no late work will be considered
towards your grades for this software project.

\begin{enumerate}

\setlength{\itemsep}{0in}

\item A properly documented, well-formatted, and (sometimes partially) completed
  version of \assessment{} that meets all of the established requirements and
  includes an assessment of a student's mastery of professional and technical
  skills in the field of software engineering.

\item A properly documented, well-formatted, and completed version of \conduct{}
  containing a full code of conduct and a statement that the student adhered to
  it during this project.

\item Available for download from a GitHub organization, a suite of seven
  separate Pytest plugins that are written in Python, integrated into Pytest,
  and in regular production use.

\end{enumerate}

\end{document}
